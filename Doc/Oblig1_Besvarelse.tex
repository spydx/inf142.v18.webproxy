\documentclass[norsk,a4paper]{article}
\usepackage[norsk]{babel}
\usepackage[utf8]{inputenc}
\usepackage{parskip}
\usepackage{amsmath}
\usepackage{amssymb}
\usepackage{csquotes}% Recommended
\usepackage{lastpage}
\usepackage{fancyhdr}
\usepackage{graphicx}

\pagestyle{fancy}
\fancyhf{}

\chead{INF142 - Oblig 1 - Oppgave 1}
\title{WebProxyServer UDP Protokoll beskrivelse}
\author{Kenneth Fossen @nix007}
\lfoot{}
\cfoot{Side \thepage{} av \pageref{LastPage}}

\begin{document}
\maketitle
\tableofcontents{}
\listoffigures

\clearpage

\section{Innledning}

Oppgaven beskriver at vi skal lage en WebProxyServer (WPS) der all kommunikasjon mellom klienten (K) og WPS blir håndtert via UDP.

I fra oppgaveteksten:
\textit{K skal kommuniserer med WPS gjennom en egen applikasjonsprotokoll som dere skal beskrive som en egen deloppgave. Denne protokollen skal dere implemterer gjennom at K og WPS kommuniserer ved hjelp av UDP/DatagramSocketer.}

Videre står det at: \textit{K skal sende \textbf{vertsnavn} og eventuelt \textbf{stinavn} til WPS. Beskriv spesielt hvordan protkollen håndterer situsajonen med at det er valgfritt for K å sende stinavn.}

Beskrivelse:
\begin{itemize}
  \item WPS skal bruke TCP for å hente kun \textit{headerinfo} med RAW HTTP fra en webserver, angitt med bruker input i fra K.
  K skal også kunne oppgi stinavn som skal hentes på server.
  Dersom K ikke oppgir stinavn skal default \textit{/}.
  \item K skal kommunisere med WPS gjennom egen applikasjonsprotokoll.
  Gjøres ved hjelp av UDP/DatagramSocketer.
\end{itemize}

\clearpage

\section{Protokoll beskrivelse for kommunikasjon med WPS}
\subsection{Kommunikasjon}

\subsubsection{Klienten}
\textbf{Kommandoer}
\begin{itemize}
  \item \textbf{GET} - henter en websti
  \item \textbf{?} - viser hjelpemeny
  \item \textbf{QUIT} - avslutter applikasjonen
\end{itemize}

Klienten er satt opp til å kommunisere med WPS Serveren via port UDP/8080.
Denne må være åpnet for i brannmur for at kommunikasjonen skal fungere mellom klient og server.

\textbf{GET} : brukes på en av de følgende måte:
\textit{GET www.uib.no}
\textit{GET www.uib.no/matnat}
\textit{GET http://www.bt.no}
\textit{GET https://www.bt.no/}

Dette vil returnere headerinfo i fra \textit{websiden} og vise det på klienten. Serveren vil da lage en HEAD post som blir sendt på enten HTTP eller HTTPS til serveren.

Klienten kan så requeste en ny eller avslutte applikasjonen sin med kommandoen \textbf{QUIT} eller \textbf{Q}.

\begin{figure}[ht!]
  \includegraphics[width=0.95\linewidth]{img/k_menu.png}
  \caption{Klient meny}
  \label{fig:klient_meny}
\end{figure}

\clearpage

\subsection{Vertnavn håndtering og Stihåndtering}
Serveren tar i mot en request på følgende format:

\textit{«GET URL»}

\textit{«GET www.kefo.no»} eller \textit{«GET www.kefo.no/wp-admin»}

Når servern mottar denne GET kommandoen blir GET og eventuelt protokoll spesifikasjon HTTP eller HTTPS strippet fra hele kommandoen, slik at en står igjen med URLen.
URLen blir ACKet i en pakke som ser slik ut ACK URL.
I dette tilfellet «ACK WWW.KEFO.NO» og vises på serverconsolet.

Så blir URL dataene fra kommandoen behandlet i processURL().
Dette gjør at når vi skal sende HTTP HEAD post over TCP så kan vi konstruere den etter HTTP standarden.

I getHttpHeaders() blir det også tatt et protokollvalg,
her bestemmes det om det skal åpnes en HTTP eller HTTPS tilkobling til websiden.

Vi følger \textit{HTTP/1.1} standarden.
Derfor blir Headeren \textit{Connection: close} lagt til siden vi ikke skal bruke \textit{persistent connections}.

For en helt vanlig sti 1. «www.kefo.no» eller 2. «kefo.no/» så blir det slik brukeren skriver.
\begin{itemize}
  \item I 1. tilfellet fil en få HTTP/1.1 301/Moved Permanently
  \item I 2. tilflelet vil en få HTTP/1.1 200 OK
\end{itemize}

Se Figur \eqref{fig:k_kefo}

\begin{figure}[ht!]
  \includegraphics[width=1\linewidth]{img/k_kefo.png}
  \caption{Klient henter www.kefo.no og kefo.no/}
  \label{fig:k_kefo}
\end{figure}

WPS Servern vil kunne håndtere følgende situasjoner:
\begin{itemize}
  \item webside med sti
  \item websiden uten sti
  \item HTTP og HTTPS om spesifisert av bruker
  \item ugyldige web addresser (FQDN)
\end{itemize}

WPS Serveren begrenser også til 5 tegn på en webadresse e.g ba.no.

Når serveren skal sende HTTP HEAD til en webside uten sti,
så blir requesten formet slik: \textit{HEAD / HTTP/1.1}

Når serveren skal sende HTTP HEAD til en webside med sti,
så blir requesten formet slik: \textit{HEAD http://FQDN/path HTTP/1.1}
og for HTTPS blir den slik: \textit{HEAD https://FQDN/path HTTP/1.1}.

\clearpage
\subsection{Feilhåndtering}

Alle feilkoder i fra HTTP protokollen blir returnert som meldinger til brukeren.

Feil, e.g FQDN feil, så vil WPS Serveren sende en RST pakke til klienten og denne vil så avbryte hentingen og kan gjøre et nytt forsøk.

\subsection{Protokoll beskrivelse}

Siden vi kjører UDP mellom klienten,
så er det ikke noen avklaring i fra serveren om at den er klar, dette medfører at klienten henger til den får kontakt med serveren.
Det er ikke noe Timeout i klienten.

Dette er følgende kommunikasjon som er for en GET request i fra klienten:
\begin{itemize}
  \item K   : get www.webside.no to WPS
  \item WPS : ack to K
  \item K   : Waiting for headers from WPS
  \item WPS : TCP request www.webside.no from WEB
  \item WEB : TCP response to WPS
  \item WPS : Headerdata to K
\end{itemize}
Det er likt for HTTP og HTTPS.

Dette vil se slik ut på en klient:
\begin{figure}[ht!]
  \includegraphics[width=0.95\linewidth]{img/k_get.png}
  \caption{Klient get request}
  \label{fig:klient_get}
\end{figure}

Dersom det er feil FQDN ,så vil serveren sende en RST istede for ACK for datane og en kan spørre på nytt i fra klienten.

\subsection{Skisse av kommunikasjon}

\begin{figure}[ht!]
  \includegraphics[width=1\linewidth]{img/k_wps_com.jpg}
  \caption{Klient WPS Kommunikasjon}
  \label{fig:k_wps_com}
\end{figure}


\end{document}
